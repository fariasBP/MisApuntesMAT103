% carpeta donde estan la imagenes
\graphicspath{{img/}}
%
%definiendo colores
\definecolor{fuxia}{RGB}{230, 0, 126}
\definecolor{primary}{RGB}{235, 91, 10}
\definecolor{secondary}{RGB}{121, 54, 163}
%
%funte tipografica
\sectionfont{\color{primary}}
\subsectionfont{\color{secondary}}
%
% caja de ejmplo

%
% caja de definicion
\tcbuselibrary{theorems, breakable, skins}
\newtcbtheorem{Theorem}{}%
{
	enhanced, % tcolorbox styles
	%attach boxed title to top center={yshift=-2.5mm},
	colback=white, colframe=fuxia, colbacktitle=white, toprule=0.2mm, leftrule=0.2mm, rightrule=0.2mm, bottomrule=0.2mm , coltitle=fuxia,
	boxed title style={size=small,colframe=white},
	%fonttitle=\bfseries,
	%rounded corners=all,
	toptitle=1ex, top=0.5ex, % a little extra space at top, a little less before content
	titlerule=-1ex, % get rid of separator rule
	title={#1},
	%fontupper=\itshape, % make theorem content italics
	%description delimiters parenthesis, % parentheses around theorem title
	description font=\normalfont,% no bold for theorem title
	separator sign none,% no punctuation after theorem name
	breakable
}{th}
% caja de ejemplo
\newtcbtheorem{Example}{}%
{
	enhanced, % tcolorbox styles
	attach boxed title to top left={yshift=-2.5mm},
	colback=white, colframe=lightgray, colbacktitle=white, toprule=0mm, leftrule=0.8mm, rightrule=0mm, bottomrule=0mm , coltitle=darkgray,
	boxed title style={size=small,colframe=white},
	%fonttitle=\bfseries,
	%rounded corners=all,
	toptitle=1ex, top=0.5ex, % a little extra space at top, a little less before content
	titlerule=-1ex, % get rid of separator rule
	title={#1},
	%fontupper=\itshape, % make theorem content italics
	%description delimiters parenthesis, % parentheses around theorem title
	description font=\normalfont,% no bold for theorem title
	separator sign none,% no punctuation after theorem name
	breakable
}{th}
%
% configurando multicols
\NewEnviron{auxmulticols}[1]{%
	\ifnum#1<2\relax% Fewer than 2 columns
	%\vspace{-\baselineskip}% Possible vertical correction
	\BODY
	\else% More than 1 column
	\begin{multicols}{#1}
		\BODY
	\end{multicols}%
	\fi
}
%
%  definiir seno
\def\sen{\mathop{\mbox{\normalfont sen}}\nolimits}
\def\ctg{\mathop{\mbox{\normalfont ctg}}\nolimits}
\def\tg{\mathop{\mbox{\normalfont tg}}\nolimits}

%para enumeraciones
\renewcommand{\theenumi}{\roman{enumi})}