\subsection*{Combinaciones lineales}
La combinacion lineal es un concepto que se ha visto  anteriormente, como en fisica basica, por ejemplo:
sea el vector $a=$
\begin{figure}[H]
	\centering
	\begin{tikzpicture}[scale=1.2]
		\begin{axis}[xmin=-5,xmax=5,ymin=-1,ymax=7,axis x line=center, axis y line=center, yticklabel=\empty, xticklabel=\empty, xlabel=$x$, ylabel=$y$, xlabel style={at={(ticklabel* cs:1.05)}}, ylabel style={at={(ticklabel* cs:1.05)}}]
			\addplot [color=red!80,thick,samples=200]{4};
			\node at (axis cs:3,4.5) {$f(x)=b$};
		\end{axis}
	\end{tikzpicture}
\end{figure}
Sea $V$ un espacio vectoriales, $\vec{x}=(x_1,x_2,x_3,\dots,x_n)$ un elemento de dicho conjunto y un subconjunto $W={\vec{y},\vec{z},\dots,\vec{w}}$. Entonces $\vec{x}$ es combinacion lineal de $W$ si se comprueba:
$$ \vec{x}=a_1\vec{y}+a_2\vec{z}+\dots+a_3\vec{w} $$
\begin{Example*} {Combinacion lineal - ejemplo 1}
	En $\mathbb{R}^3$, escribir $\vec{x}=(1,2,3)$ como combinacion lineal de $\vec{u}=(1,0,1)$ $\vec{v}=(2,-1,1)$ y $\vec{w}=(1,1,1)$
	Sol.
	\begin{align*}
		&\vec{x}=\alpha\vec{u}+\beta\vec{v}+\gamma\vec{w}\\
		&(1,2,3)=\alpha(1,0,1)+\beta(2,-1,1)+\gamma(1,1,1)\\
		&(1,2,3)=(\alpha,0,\alpha)+(2\beta,-\beta,\beta)+(\gamma,\gamma,\gamma)\\
		&(1,2,3)=(\alpha+2\beta+\gamma,-\beta+\gamma,\alpha+\beta+\gamma)\\
		&\left\{\begin{array}{r}
			\alpha+2\beta+\gamma=1\\
			-\beta+\gamma=2\\
			\alpha+\beta+\gamma=3
		\end{array}\right.\\
		&\begin{bmatrix}
			1&2&1\\
			0&-1&1\\
			1&1&1
		\end{bmatrix}\begin{bmatrix}
			\alpha\\
			\beta\\
			\gamma
		\end{bmatrix}=\begin{bmatrix}
		 	1\\2\\3
		\end{bmatrix}\\
		&\begin{array}{c}
			\left[\begin{array}{ccc|c}
				1&2&1&1\\
				0&-1&1&2\\
				1&1&1&3
			\end{array}\right]\\
			-f_1+f_3\rightarrow f_3
		\end{array}\begin{array}{c}
			\left[\begin{array}{ccc|c}
				1&2&1&1\\
				0&-1&1&2\\
				0&-1&0&2
			\end{array}\right]\\
			-f_2\rightarrow f_2
		\end{array}\\
		&\begin{array}{c}
			\left[\begin{array}{ccc|c}
				1&2&1&1\\
				0&1&-1&-2\\
				0&-1&0&2
			\end{array}\right]\\
			-2f_2+f_1\rightarrow f_1\\
			f_2+f_3\rightarrow f_3
		\end{array}\begin{array}{c}
			\left[\begin{array}{ccc|c}
				1&0&3&5\\
				0&1&-1&-2\\
				0&0&-1&0
			\end{array}\right]\\
			-f_3\rightarrow f_3\\ \
		\end{array}\\
		&\begin{array}{c}
			\left[\begin{array}{ccc|c}
				1&0&3&5\\
				0&1&-1&-2\\
				0&0&1&0
			\end{array}\right]\\
			-3f_3+f_1\rightarrow f_1\\
			f_3+f_2\rightarrow f_2
		\end{array}\begin{array}{c}
			\left[\begin{array}{ccc|c}
				1&0&0&5\\
				0&1&0&-2\\
				0&0&1&0
			\end{array}\right]\\
				\ \\ \
		\end{array}\\
		&\begin{matrix}
			\alpha=5&\beta=-2&\gamma-0
		\end{matrix}\\
		&\text{(tienen uncia solución)}\\
		&\therefore \vec{x} \text{ es un C.L. de } \vec{u}, \vec{v} \text{ y } \vec{w}\\
		&\vec{x}=5\vec{u}+(-2)\vec{v}+(0)\vec{w}\\
	\end{align*}
\end{Example*}