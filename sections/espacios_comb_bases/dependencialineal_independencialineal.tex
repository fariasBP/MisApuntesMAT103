\subsection*{Dependencia e independencia lineal}
\begin{Theorem*} {Dependencia lineal}
	Sea $V$ un espacio vectorial sobre un cuerpo $K$. Se dice que los vectores $v_1,v_2,\dots,v_n \ \in V$ son linealmente independientes (LI) sobre $K$ si existe escalares no todos $0$ tal que se cumpla la siguiente combinacion lineal:
	$$ k_1\vec{u}_1+k_2\vec{u}_2+k_3\vec{u}_3+\dots+k_n\vec{u}_n $$
	En caso contrario se dice qu los vectores son linealmente dependientes sobre $K$.
\end{Theorem*}
Es decir que un subconjunto de vectores son linealmente independientes si no tienen una misma dirección, así también son linealmente dependientes si la tienen.
\begin{Example*} {LI - ejemplo 1}
	analizar si $x^2+3x+2$,$2x^2-1$ y $3+2x-x^2$ son linealmente independientes.
	Sol.
	\begin{align*}
		&p(x)=x^2+3x+2\\
		&q(x)=2x^2-1\\
		&r(x)=-x^2+2x+3\\
		&k_1p(x)+k_2q(x)+k_3r(x)=0x^2+0x+0\\
		&k_1(x^2+3x+2)+k_2(2x^2-1)+k_3(-x^2+2x\\
		&+3)=0x^2+0x+0\\
		&\left\{\begin{array}{r}
			k_1+2k_2-k_3=0\\
			3k_1+2k_3=0\\
			2k_1+k_2+3k_3=0
		\end{array}\right.\\
		&\begin{bmatrix}
			1&2&-1\\
			3&0&2\\
			2&-1&3
		\end{bmatrix}\begin{bmatrix}
			k_1\\k_2\\k_3
		\end{bmatrix}=\begin{bmatrix}
			0\\0\\0
		\end{bmatrix}\\
		&\begin{array}{r}
			|A|=\begin{vmatrix}
				1&2&-1\\
				3&0&2\\
				2&-1&3
			\end{vmatrix}\\
			2f_3+f_1\rightarrow f_1
		\end{array}\begin{array}{c}
			=\begin{vmatrix}
				5&0&5\\
				3&0&2\\
				2&-1&3
			\end{vmatrix}\\
			\
		\end{array}\\
		&=0+0+(-1)(-1)^{3+2}\begin{vmatrix}
			1&-1\\
			3&2
		\end{vmatrix}=-5\\
		&|A|\ne 0 \Rightarrow \text{es LI}\\
		&\therefore \text{los polinomios son linealmente independientes}
	\end{align*}
\end{Example*}