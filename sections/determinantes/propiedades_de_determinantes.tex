El determinante se define como:
\begin{Theorem*} {Determinante de un matriz}
	Sea una matriz cuadrada $ A=[ a_{ij} ] $, entonces existe un escalar particular denominado \textit{determinante} denotado por $\mathrm{det}(A)$ ó $|A|$.
\end{Theorem*}
Es decir un valor que representa a una matriz y esta debe ser único, no cambia si se obtiene por uno u otro método.
\subsection{Propiedades de determinantes}
\subsubsection*{Valores nulos}
\begin{enumerate}
	\item Un determinante con una fila o columna compuestas por ceros, tiene un valor de cero:
	\begin{align*}
		&\begin{vmatrix}
			1&2&3\\
			0&0&0\\
			4&5&6\\
		\end{vmatrix}=0\quad\quad\begin{vmatrix}
			1&2&0\\
			3&4&0\\
			5&6&0
		\end{vmatrix}=0
	\end{align*}
	\item Un determinante con dos filas o columnas iguales es cero
	\begin{align*}
		&\begin{vmatrix}
			1&3&5\\
			0&-4&6\\
			1&3&5\\
		\end{vmatrix}=0\quad\quad\begin{vmatrix}
			1&2&1\\
			0&3&0\\
			9&7&9
		\end{vmatrix}=0
	\end{align*}
	\item Un determinante con una fila o columna que sea múltiplo de otra fila o columna es cero
	\begin{align*}
		&\begin{vmatrix}
			1&2&3\\
			4&5&6\\
			3&6&9\\
		\end{vmatrix}=0\quad\quad\begin{vmatrix}
			2&14\\
			7&49
		\end{vmatrix}=0
	\end{align*}
	note usted que la $f_3$ es múltiplo de $f_1$ en la primera determinante y $c_1$ es múltiplo de $c_2$ en la segunda determinante.
\end{enumerate}
\subsubsection*{Valores notables}
\begin{enumerate}
	\item El determinante de una matriz identidad es 1
	\begin{align*}
		&\begin{vmatrix}
			1&0&0\\
			0&1&0\\
			0&0&1\\
		\end{vmatrix}=1
	\end{align*}
	\item El determinante de la inversa de una matriz por el determinante de la matriz es igual a 1.
	\begin{align*}
		|A^{-1}||A|=1 \quad\quad |A||A^{-1}|=1
	\end{align*}
	\item El determinante de una matriz tirangular es igual al producto de sus elemento de la diagonal principal:
	\begin{align*}
		&\begin{vmatrix}
			a_{11}&a_{12}&a_{13}\\
			0&a_{22}&a_{23}\\
			0&0&a_{33}\\
		\end{vmatrix}=a_{11}a_{22}a_{33}
	\end{align*}
\end{enumerate}
\subsubsection*{Operaciones elementales}
Supongamos que B se ha obtenido de A mediante una operación elemental entre filas y/o columnas por lo tanto:
\begin{enumerate}
	\item Si se han intercambiado filas o columnas de A, B queda multiplicado por un uno negativo.
	$$ |A|=-|B| $$
	por ejemplo:
	\begin{align*}
		&\begin{matrix}
			\begin{vmatrix}
				3&-1&0\\
				5&1&2\\
				3&8&-2\\
			\end{vmatrix}=\\
			f_1\leftrightarrow f_2
		\end{matrix} \begin{matrix}
			\ -\begin{vmatrix}
				5&1&2\\
				3&-1&0\\
				3&8&-2\\
			\end{vmatrix}\\
			\
		\end{matrix}
	\end{align*}
	\item Si se ha multiplicado una fila o columna de A por un escalar k, entonces se multiplica a B por la inversa del escalar.
	$$ |A|=\frac{1}{k}|B| $$
	por ejemplo:
	\begin{align*}
		&\begin{matrix}
			\begin{vmatrix}
				5&4&1\\
				3&0&-1\\
				7&-9&2
			\end{vmatrix}= \ \\
			1/5 f_1\rightarrow f_2
		\end{matrix}\begin{matrix}
		5\begin{vmatrix}
			1&4/5&1/5\\
			3&0&-1\\
			7&-9&2
		\end{vmatrix}\\
		\
		\end{matrix}
	\end{align*}
	esta propiedad es muy útil ya que lo podemos usar como factorización, sin embargo note que solo lo factorizamos a una fila o a una columna:
	\begin{align*}
		&\begin{vmatrix}
			1/2&4/3&1/5\\
			3&0&-1\\
			7/6&-9/3&2/12
		\end{vmatrix}=\frac{1}{3}\begin{vmatrix}
		1/2&4&1/5\\
		3&0&-1\\
		7/6&-9&2/12
		\end{vmatrix}\\
		&\begin{vmatrix}
			2&4&24\\
			3&3&7\\
			1&0&-1
		\end{vmatrix}=2\begin{vmatrix}
			1&2&12\\
			3&3&7\\
			1&0&-1
		\end{vmatrix}
	\end{align*}
	\item Si se ha sumado un múltiplo de una fila o columna a otra en A, no se agrega nada adicional a B.
	$$ |A|=|B| $$
	por ejemplo
	\begin{align*}
		&\begin{array} {c}
			\begin{vmatrix}
				1&5&-1\\
				8&1&0\\
				-2&3&7
			\end{vmatrix} \ \\
			-8f_1+f_2\rightarrow f_2
		\end{array}\begin{matrix}
		= \ \ \begin{vmatrix}
			1&5&-1\\
			0&-40&8\\
			-2&3&7
		\end{vmatrix}\\
			\
		\end{matrix}
	\end{align*}
\end{enumerate}