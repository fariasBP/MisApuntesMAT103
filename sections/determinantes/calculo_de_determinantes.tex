\subsection*{Calculo de determinantes I}
Existen varios métodos para la resolución de determinantes de una matriz:
\subsubsection*{Método de Sarrus}
Este método consiste en multiplicar los elementos de una digonal de baja restada con el producto de los elementos de una diagonal de subida, de la siguiente forma:
\begin{center}
	\begin{tikzpicture}
		\matrix (M) [matrix of math nodes , inner sep=1pt, row sep=0.6cm,column sep=8mm]{
			&&& -\\      
			a_{11} & a_{12} & a_{13} & - \\
			a_{21} & a_{22} & a_{23} & -\\ 
			a_{31} & a_{32} & a_{33} & \ \\
			a_{11} & a_{12} & a_{13} & + \\
			a_{21} & a_{22} & a_{23} & +\\
			&&& +\\  
		};
		
		
		\draw (M-2-1.north west) -- (M-4-1.south west) ;
		\draw (M-2-3.north east) -- (M-4-3.south east) ;
		
		\draw[blue](M-2-1)--(M-3-2)--(M-4-3)--(M-5-4);
		\draw[blue](M-3-1)--(M-4-2)--(M-5-3)--(M-6-4);
		\draw[blue](M-4-1)--(M-5-2)--(M-6-3)--(M-7-4);
		
		\draw[red](M-5-1)--(M-4-2)--(M-3-3)--(M-2-4);
		\draw[red](M-4-1)--(M-3-2)--(M-2-3)--(M-1-4);
		\draw[red](M-6-1)--(M-5-2)--(M-4-3)--(M-3-4);
	\end{tikzpicture}
\end{center}
o también:
\begin{center}
	\begin{tikzpicture}
		\matrix (M) [matrix of math nodes , inner sep=1pt, row sep=0.6cm,column sep=5mm]{
			&&& - & - & -\\      
			a_{11} & a_{12} & a_{13} & a_{11} & a_{12} & \ \\
			a_{21} & a_{22} & a_{23} & a_{21} & a_{22} & \ \\ 
			a_{31} & a_{32} & a_{33} & a_{31} & a_{32} & \ \\
			&&& + & + & +\\  
		};
		
		\draw (M-2-1.north west) -- (M-4-1.south west) ;
		\draw (M-2-3.north east) -- (M-4-3.south east) ;
		
		\draw[blue](M-2-1)--(M-3-2)--(M-4-3)--(M-5-4);
		\draw[blue](M-2-2)--(M-3-3)--(M-4-4)--(M-5-5);
		\draw[blue](M-2-3)--(M-3-4)--(M-4-5)--(M-5-6);
		
		\draw[red](M-4-1)--(M-3-2)--(M-2-3)--(M-1-4);
		\draw[red](M-4-2)--(M-3-3)--(M-2-4)--(M-1-5);
		\draw[red](M-4-3)--(M-3-4)--(M-2-5)--(M-1-6);
	\end{tikzpicture}
\end{center}
Cabe resaltar que \textbf{este método es solo valido hasta una matriz de 3x3}.
\begin{Example*} {Método Sarrus para determinantes}
	Hallar el determinante de:
	$$
		A=\begin{bmatrix}
			1&3\\
			8&5
		\end{bmatrix}\text{ y } B=\begin{bmatrix}
			3&5&1\\
			0&-2&4\\
			1&-1&7
		\end{bmatrix}
	$$
	Sol:
	\begin{align*}
		|A|=&\begin{vmatrix}
				1&3\\
				8&5
			\end{vmatrix}\\
		=&(1)(5)-(8)(3)\\
		|A|=-19\\
		|B|=&\begin{array} {c}
			\begin{vmatrix}
				3&5&1\\
				0&-2&4\\
				1&-1&7
			\end{vmatrix}\\
			\begin{matrix}
				3&5&1\\
				0&-2&4
			\end{matrix}
		\end{array}\\
		=&(3)(-2)(7)+(0)(-1)(1)+(1)(5)(4)\\
		&-(1)(-2)(1)-(3)(-1)(4)-(0)(5)(7)\\
		|B|=-8
	\end{align*}
\end{Example*}
\subsubsection*{Método de operaciones elementales}
Este método consiste en transformar un determinante a una forma triangular superior o inferior, por medio de operaciones elementales y la propiedades de operaciones elementales-valores notables.
Un ventaja a diferencia del método Sarrus es que no importa el tamaño de la matriz.
\begin{align*}
	&\begin{array} {c}
		\begin{vmatrix}
			1&1&-2\\
			-1&2&1\\
			0&1&-1
		\end{vmatrix}\\
		f_1+f_2\rightarrow f_2
	\end{array}\begin{array} {c}
	=\begin{vmatrix}
		1&1&-2\\
		0&3&-1\\
		0&1&-1
	\end{vmatrix}\\
		1/3f_2\rightarrow f_2
	\end{array}	\begin{array} {c}
	=3\begin{vmatrix}
		1&1&-2\\
		0&1&-1/3\\
		0&1&-1
	\end{vmatrix}\\
		-f_1+f_3\rightarrow f_3
	\end{array}\\
	&3\begin{vmatrix}
		1&1&-2\\
		0&1&-1/3\\
		0&0&-2/3
	\end{vmatrix}=3\left(-\frac{2}{3}\right)=-2
\end{align*}