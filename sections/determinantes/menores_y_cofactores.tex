\subsection*{Menores y cofactores}
\subsubsection*{Menores}
Sea una matriz $n$-cuadrada $A=[a_{ij})$ y la submatriz $(n-1)$-cuadrada de A denotada por $M_{ij}$, que se obtine al suprimir $a_{ij}$, su $i$-ésima fila y su $j$-ésima columna. Entonces denominamos \textbf{menor} del elemento $a_{ij}$ de $A$ al determinante $|M_{ij}|$.
\begin{Example*} {Menores}
	Hallar el menor $|M_{11}|$, $|M_{22}|$ y $|M_{32}|$ de:
	$$ A=\begin{bmatrix}
		3&-1&2\\
		-1&9&3\\
		3&5&2
	\end{bmatrix} $$
	Sol. \\
	\begin{tikzpicture}
		\matrix (M) [matrix of math nodes , left delimiter={[},right delimiter={]} , inner sep=1pt, row sep=1.1mm,column sep=1.6mm]{    
			\ 3 & -1 & \ 2\\
			-1 & \ 9 & \ 3\\
			\ 3 & \ 5 & \ 2 \\
		};
		\draw [red] (M-1-1.north) -- (M-3-1.south) ;
		\draw [red] (M-1-1.west) -- (M-1-3.east) ;
	\end{tikzpicture}
	\begin{align*}
		&|M_{11}|=\begin{vmatrix}
			9&3\\
			5&2
		\end{vmatrix}=45-48=-3 \quad\quad\quad\quad
	\end{align*}
	\begin{tikzpicture}
		\matrix (M) [matrix of math nodes , left delimiter={[},right delimiter={]} , inner sep=1pt, row sep=1.1mm,column sep=1.6mm]{    
			\ 3 & -1 & \ 2\\
			-1 & \ 9 & \ 3\\
			\ 3 & \ 5 & \ 2 \\
		};
		\draw [red] (M-1-2.north) -- (M-3-2.south) ;
		\draw [red] (M-2-1.west) -- (M-2-3.east) ;
	\end{tikzpicture}
	\begin{align*}
		&|M_{22}|=\begin{vmatrix}
			3&2\\
			3&2
		\end{vmatrix}=6-6=0 \quad\quad\quad\quad
	\end{align*}
	\begin{tikzpicture}
		\matrix (M) [matrix of math nodes , left delimiter={[},right delimiter={]} , inner sep=1pt, row sep=1.1mm,column sep=1.6mm]{    
			\ 3 & -1 & \ 2\\
			-1 & \ 9 & \ 3\\
			\ 3 & \ 5 & \ 2 \\
		};
		\draw [red] (M-1-2.north) -- (M-3-2.south) ;
		\draw [red] (M-3-1.west) -- (M-3-3.east) ;
	\end{tikzpicture}
	\begin{align*}
		&|M_{32}|=\begin{vmatrix}
			3&2\\
			-1&3
		\end{vmatrix}=9-(-2)=11 \quad\quad\quad\quad
	\end{align*}
\end{Example*}
Así también podemos hallar los menores de cada elemento de una matriz, y lo llamaremos matriz de menores.
\begin{Example*} {Matriz de menores}
	Hallar matriz de menores $M$ si:
	$$ A=\begin{bmatrix}
		9&0&-1\\
		3&5&2\\
		-7&3&1
	\end{bmatrix} $$
	Sol.
	\begin{align*}
		&A=\begin{bmatrix}
			9&0&-1\\
			3&5&2\\
			-7&3&1
		\end{bmatrix}\\
		&M=\begin{bmatrix}
			M_{11}&M_{12}&M_{13}\\
			M_{21}&M_{22}&M_{23}\\
			M_{31}&M_{32}&M_{33}
		\end{bmatrix}\\
		&M=\begin{bmatrix}
			\begin{vmatrix}
				5&2\\
				3&1
			\end{vmatrix}&\begin{vmatrix}
				3&2\\
				-7&1
			\end{vmatrix}&\begin{vmatrix}
				3&5\\
				-7&3
			\end{vmatrix}\\
			\begin{vmatrix}
				0&-1\\
				3&1
			\end{vmatrix}&\begin{vmatrix}
				9&-1\\
				-7&1
			\end{vmatrix}&\begin{vmatrix}
				9&0\\
				-7&3
			\end{vmatrix}\\
			\begin{vmatrix}
				0&-1\\
				5&2
			\end{vmatrix}&\begin{vmatrix}
				9&-1\\
				3&2
			\end{vmatrix}&\begin{vmatrix}
				9&0\\
				3&5
			\end{vmatrix}
		\end{bmatrix}\\
		&\therefore \ M=\begin{bmatrix}
			-1&17&44\\
			3&2&27\\
			5&21&45
		\end{bmatrix}
	\end{align*}
\end{Example*}
\subsubsection*{Cofactores}
Sea una matriz cuadrada $A=[a_{ij}]$ y $|M_{ij}|$ el menor del elemento $a_{ij}$; entonces definimos el cofactor de $a_{ij}$, denotado por $A_{ij}$ como el menor con signo:
$$ A_{ij}=(-1)^{i+j}|M_{ij} $$
Es decir si que si le colocamos a un menor el signo de acuerdo a $(-1)^{i+j}$, entonces ahora es un cofactor.
\begin{Example*} {Cofactor}
	Hallar el cofactor $A_{11}$, $A_{22}$ y $A_{32}$ si:
	$$ A=\begin{bmatrix}
		3&-1&2\\
		-1&9&3\\
		3&5&2
	\end{bmatrix} $$
	Sol. \\
	\begin{align*}
		&\text{si: } |M_{11}|=-3, \ |M_{22}|=0\text{ y }|M_{32}|=11\\
		&A_{11}=(-1)^{1+1}|M_{11}|\\
		&A_{11}=(-1)^2(-3)=-3\\
		&A_{22}=(-1)^{2+2}|M_{22}|\\
		&A_{22}=(-1)^4(0)=0\\
		&A_{32}=(-1)^{3+2}|M_{32}|\\
		&A_{32}=(-1)^5(11)=-11\\
	\end{align*}
\end{Example*}
Así también podemos hallar los cofactores de cada elemento de una matriz, y lo llamaremos matriz de cofactores. Sin embargo, note usted que los signos que acompañan a los menores se disponen en forma de tablero de ajedrez, con signos $+$ en la diagonal principal:
$$ \begin{bmatrix}
	+&-&+&-&\cdots\\
	-&+&-&+&\cdots\\
	+&-&+&-&\cdots\\
	-&+&-&+&\cdots\\
	\vdots&\vdots&\vdots&\vdots&\ddots\\
\end{bmatrix} $$
Esto resulta muy útil ya que no habrá la necesidad de hallar el signo del menor de la anterior manera.
\begin{Example*} {Matriz de cofactores}
	Hallar matriz de cofactores $C$ si:
	$$ A=\begin{bmatrix}
		9&0&-1\\
		3&5&2\\
		-7&3&1
	\end{bmatrix} \text{ y } M=\begin{bmatrix}
	-1&17&44\\
	3&2&27\\
	5&21&45
	\end{bmatrix} $$
	donde $M$ es la matriz de menores\\
	Sol.
	\begin{align*}
		&\text{si: } M=\begin{bmatrix}
			-1&17&44\\
			3&2&27\\
			5&21&45
		\end{bmatrix} \text{ y } \begin{bmatrix}
			+&-&+\\
			-&+&-\\
			+&-&+\\
		\end{bmatrix}\\
		&\therefore \ C=\begin{bmatrix}
			-1&-17&44\\
			-3&2&-27\\
			5&-21&45
		\end{bmatrix}
	\end{align*}
\end{Example*}
