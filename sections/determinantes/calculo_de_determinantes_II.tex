\subsection*{Calculo de determinantes II}
\subsubsection*{Metodo de cofactores}
Sea una matriz cuadrada $A=[a_{ij}]$ entonces su determinante es la sumatoria al producto parcial de los elementos de una fila o columna cualquiera por sus correspondientes cofactores.
$$ \det(A)=\sum_{k=1}^{n}a_{ik}C_{ik} \longrightarrow \text{ (para filas) } $$
$$ \det(A)=\sum_{k=1}^{n}a_{kj}C_{kj} \longrightarrow \text{ (para columnas) } $$
Es decir, este método consiste en elegir una fila o columna cualquiera, el cual se suma el producto de elemento por el cofacator de cada elemento.
\begin{Example*} {Método de cofactores - ejemplo 1}
	Hallar la determinante de:
	$$ A=\begin{bmatrix}
		3&5&-1\\
		2&-3&6\\
		7&0&4
	\end{bmatrix} $$
	Sol.\\
	Seleccionar una fila o columna
	\begin{center}
		\begin{tikzpicture}
			\matrix (M) [matrix of math nodes , left delimiter={[},right delimiter={]} , inner sep=1pt, row sep=1.1mm,column sep=1.6mm]{    
				3 & 5 & -1\\
				2 & -3 & 6\\
				7 & 0 & 4\\
			};
			\draw [red] (M-1-1.south west) -- (M-1-3.south east);
		\end{tikzpicture}
	\end{center}
	hallamos sus cofactores
	\begin{center}
		\begin{tikzpicture}
			\matrix (M) [matrix of math nodes , left delimiter={[},right delimiter={]} , inner sep=0.5pt, row sep=1.1mm,column sep=1.6mm]{    
				3 & 5 & -1\\
				2 & -3 & 6\\
				7 & 0 & 4\\
			};
			\draw [red] (M-1-1.north) -- (M-3-1.south);
			\draw [red] (M-1-1.west) -- (M-1-3.east);
		\end{tikzpicture}
		\begin{tikzpicture}
			\matrix (M) [matrix of math nodes , left delimiter={[},right delimiter={]} , inner sep=0.5pt, row sep=1.1mm,column sep=1.6mm]{    
				3 & 5 & -1\\
				2 & -3 & 6\\
				7 & 0 & 4\\
			};
			\draw [red] (M-1-2.north) -- (M-3-2.south);
			\draw [red] (M-1-1.west) -- (M-1-3.east);
		\end{tikzpicture}
		\begin{tikzpicture}
			\matrix (M) [matrix of math nodes , left delimiter={[},right delimiter={]} , inner sep=0.5pt, row sep=1.1mm,column sep=1.6mm]{    
				3 & 5 & -1\\
				2 & -3 & 6\\
				7 & 0 & 4\\
			};
			\draw [red] (M-1-3.north) -- (M-3-3.south);
			\draw [red] (M-1-1.west) -- (M-1-3.east);
		\end{tikzpicture}
	\end{center}
	\begin{align*}
		&A_{11}=-12 \quad\quad A_{12}=-34 \quad\quad A_{13}=21
		\intertext{hallamos su determinante}
		&|A|=\begin{vmatrix}
			3&5&-1\\
			2&-3&6\\
			7&0&4
		\end{vmatrix}\\
		&=a_{11}(-1)^2A_{11}+a_{12}(-1)^3A_{12}+a_{13}(-1)^4A_{13}\\
		&=(3)(-12)+(5)(-1)(-34)+(-1)(21)\\
		&=113\\
		&\therefore \ |A|=113
	\end{align*}
\end{Example*}
Este metodo es mas util para matrices con fila o columnas que tengas elementos nulos (ceros) ya que al multiplicar por elemento esta se anula:
\begin{Example*} {Metodo de cofactores - ejemplo 2}
	Hallar la determinante de:
	$$ A=\begin{bmatrix}
		0&0&-2&3&1&2\\
		1&-1&2&-1&0&-3\\
		3&0&1&-1&-2&4\\
		5&0&-4&3&1&1\\
		-2&0&2&9&-9&1\\
		8&0&-3&1&-1&0
	\end{bmatrix} $$
	Sol.
	\begin{align*}
		|A|=(-1)\begin{vmatrix}
			0&-2&3&1&2\\
			3&1&-1&-2&4\\
			5&-4&3&1&1\\
			-2&2&9&-9&1\\
			8&-3&1&-1&0
		\end{vmatrix}
	\end{align*}
\end{Example*}
\subsubsection*{Método de Chio (Método combinado)}
Este método consiste en transformar por medio de operaciones elementales una determinante con el objetivo de generar ceros en fila o columna, una ves esto por el método de cofactores encontrar la determinante.
\begin{Example*} {Metodo de Chio}
	Hallar el determinante de:
	$$ A=\begin{bmatrix}
		1&234/134&\sqrt{249}\\
		-1&2&1\\
		0&1&-1
	\end{bmatrix} $$
	Sol.
	\begin{align*}
		&\begin{matrix}
			|A|=\begin{vmatrix}
				1&\frac{234}{134}&\sqrt{349}\\
				-1&2&1\\
				0&1&-1
			\end{vmatrix}\\
			f_1+f_2\rightarrow f_2
		\end{matrix} \begin{matrix}
			=\begin{vmatrix}
				1&\frac{234}{134}&\sqrt{349}\\
				0&3&-1\\
				0&1&-1
			\end{vmatrix}\\
			\
		\end{matrix}\\
		&|A|=(1)\begin{vmatrix}
			3&-1\\
			1&-1
		\end{vmatrix}-\cancel{(0)\begin{vmatrix}
			\frac{234}{134}&\sqrt{349}\\
			1&-1
		\end{vmatrix}}\\
		&\quad\quad+\cancel{(0)\begin{vmatrix}
			\frac{234}{134}&\sqrt{349}\\
			3&-1
		\end{vmatrix}}\\
		&|A|=\begin{vmatrix}
			3&-1\\
			1&-1
		\end{vmatrix}=3(-1)-(1)(-1)=-2
	\end{align*}
\end{Example*}