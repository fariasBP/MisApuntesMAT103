\end{multicols}
\section{Sistema de ecuaciones lineales}
\subsubsection*{Ecuaciones lineales}
Una ecuación lineal en las variables $ x_1, x_2, x_3,\dots , x_n $ tiene la forma:
$$ a_1x_1+a_2x_2+a_3x_3+\dots+a_nx_n=b_1 $$
donde: $ a_1,a_2,a_3,\dots,a_n,b_1 \in \mathbb{R} $. \\
por ejemplo
\begin{flalign*}
	&\begin{array} {l}
		2x+9y-3z=2\longrightarrow \text{(es ecuación lineal)}\\
		x_1-2x_2=0\longrightarrow \text{(es ecuación lineal)}\\
		x_1+x_2+\dots+x_n=1\longrightarrow \text{(es ecuación lineal)}\\
	\end{array} &\begin{array} {l}
		3xy-2y=9\longrightarrow \text{(no es ecuación lineal)}\\
		9y^2-\frac{2}{x}=-2\longrightarrow \text{(no es ecuación lineal)}\\
		x^2y-1=y\longrightarrow \text{(no es ecuación lineal)}\\
	\end{array}
\end{flalign*}
\subsubsection*{Sistema de ecuaciones lineales}
Es un conjunto finito de ecuaciones lineales
$$ \left\{\begin{matrix}
	a_{11}x_1+a_{12}x_2+a_{13}x_3+\dots+ \ a_{1n}x_n=b_1\\
	a_{21}x_1+a_{22}x_2+a_{23}x_3+\dots+ \ a_{2n}x_n=b_2\\
	a_{31}x_1+a_{32}x_2+a_{33}x_3+\dots+ \ a_{3n}x_n=b_3\\
	\vdots\\
	a_{m1}x_1+a_{m2}x_2+a_{m3}x_3+\dots+a_{mn}x_n=b_m\\
\end{matrix}\right. $$
En función a la definición de igualdad de matrices, es decir dos matrices son iguales cuando tienen el mismo tamaño y sus elementos son iguales posición a posición, entonces podemos pasar de un sistema de ecuaciones a una matricial:
$$
	\begin{bmatrix}
		a_{11}x_1+a_{12}x_2+a_{13}x_3+\dots+ \ a_{1n}x_n\\
		a_{21}x_1+a_{22}x_2+a_{23}x_3+\dots+ \ a_{2n}x_n\\
		a_{31}x_1+a_{32}x_2+a_{33}x_3+\dots+ \ a_{3n}x_n\\
		\vdots\\
		a_{m1}x_1+a_{m2}x_2+a_{m3}x_3+\dots+a_{mn}x_n\\
	\end{bmatrix}_{m\cross 1}=\begin{bmatrix}
		b_1\\
		b_2\\
		b_3\\
		\vdots\\
		b_m
	\end{bmatrix}_{m\cross 1}
$$
Ahora podemos separar los coeficientes de sus variables con el producto de matrices:
$$
	\begin{bmatrix}
		a_{11}&a_{12}&a_{13}&\cdots&a_{1n}\\
		a_{21}&a_{22}&a_{23}&\cdots&a_{2n}\\
		a_{31}&a_{32}&a_{33}&\cdots&a_{3n}\\
		\vdots&\vdots&\vdots&\ddots&\vdots\\
		a_{m1}&a_{m2}&a_{m3}&\cdots&a_{mn}\\
	\end{bmatrix}_{m\cross n}\begin{bmatrix}
		x_1\\
		x_2\\
		x_3\\
		\vdots\\
		x_n
	\end{bmatrix}_{n\cross 1}=\begin{bmatrix}
		a_{11}x_1+a_{12}x_2+a_{13}x_3+\dots+ \ a_{1n}x_n\\
		a_{21}x_1+a_{22}x_2+a_{23}x_3+\dots+ \ a_{2n}x_n\\
		a_{31}x_1+a_{32}x_2+a_{33}x_3+\dots+ \ a_{3n}x_n\\
		\vdots\\
		a_{m1}x_1+a_{m2}x_2+a_{m3}x_3+\dots+a_{mn}x_n\\
	\end{bmatrix}_{m\cross 1}
$$
Entonces
$$
	\begin{bmatrix}
		a_{11}&a_{12}&a_{13}&\cdots&a_{1n}\\
		a_{21}&a_{22}&a_{23}&\cdots&a_{2n}\\
		a_{31}&a_{32}&a_{33}&\cdots&a_{3n}\\
		\vdots&\vdots&\vdots&\ddots&\vdots\\
		a_{m1}&a_{m2}&a_{m3}&\cdots&a_{mn}\\
	\end{bmatrix}_{m\cross n}\begin{bmatrix}
		x_1\\
		x_2\\
		x_3\\
		\vdots\\
		x_n
	\end{bmatrix}_{n\cross 1}=\begin{bmatrix}
		b_1\\
		b_2\\
		b_3\\
		\vdots\\
		b_m
	\end{bmatrix}_{m\cross 1}
$$
finalmente obtenemos un sistema matricial de ecuaciones lineales, donde podemos trabajar, transforma y el cual es nuestro objetivo resolver el sistema, usando distintos propiedades matriciales que ya aprendimos:
$$ A_{m\cross n}X_{n\cross 1} = B_{m\cross 1} $$
por lo tanto:
\begin{Theorem*} {Teoría del álgebra lineal para resolución de sistema de ecuaciones lineales}
	Sea sistema de ecuaciones lineales de la forma:
	$$ a_1x_1+a_2x_2+a_3x_3+\dots+a_nx_n=b_1 $$
	se puede expresar a una forma de de ecuación matricial:
	$$ A_{m\cross n}X_{n\cross 1} = B_{m\cross 1} $$
	donde $A$ es la matriz de coeficientes que pertenecen a $\mathbb{R}$, $X$ es la matriz de variables y $B$ es la matriz de términos independientes que pertenecen a $\mathbb{R}$.
\end{Theorem*}
\subsection*{Tipos de soluciones}
\subsection*{Métodos de resolución}
\subsubsection*{Método de la inversa}
\begin{Theorem*} {Metodo de la inversa}
	Sea $A$ una \textbf{matriz $n$-cuadrada}, $B$ matriz de terminos independientes y $X$ es la matriz de variables, entonces:
	\begin{gather*}
		AX=B\\
		A^{-1}AX=A^{-1}B
	\end{gather*}
	donde:
	$$ A_{n\cross n}, |A|\ne 0, A=[a_{ij}] \in \mathbb{R}, B=[b_{ij}] \in \mathbb{R} $$
	por lo tanto:
	$$ X=A^{-1}B $$
\end{Theorem*}
\subsubsection*{Método de Gauss o forma escalonada}
\begin{Theorem*} {Método de Gauss}
	Sea $A$,$B$y$X$ la matriz de coeficientes, terminos independientes y de variables respectivamente; $U$ una matriz triangular superior y $B'$ una matriz transformada por operaciones elementales a partir de $B$. Entonces:
	\begin{gather*}
		AX=B\\
		[A|B]\xrightarrow[\text{operaciones elementales}]{\text{Transformar por}}[U|B']\\
		UX=B'
	\end{gather*}
	donde:
	$$ A=[a_{ij}] \in \mathbb{R}, B=[b_{ij}] \in \mathbb{R} $$
\end{Theorem*}
\subsubsection*{Método de Gauss-Jordan o forma escalonada reducida}
\begin{Theorem*} {Método de Gauss-Jordan}
	Sea $A$,$B$y$X$ la matriz de coeficientes, terminos independientes y de variables respectivamente; $I$ e $I'$ una matriz identidad y matriz semejante a la identidad; y $B'$ una matriz transformada por operaciones elementales a partir de $B$. Entonces:
	$$ AX=B $$
	existen dos casos:
	\begin{enumerate}
		\item se llega a la matriz identidad $I$:
		\begin{gather*}
			[A|B]\xrightarrow[\text{operaciones elementales}]{\text{Transformar por}}[I|B']\\
			IX=B'
		\end{gather*}
		\item se llega a una matriz $I'$ parecida a la identidad, debido a que $A$ no es invertible:
		\begin{gather*}
			[A|B]\xrightarrow[\text{operaciones elementales}]{\text{Transformar por}}[I'|B']\\
			I'X=B'
		\end{gather*}
	\end{enumerate}
	donde:
	$$ A=[a_{ij}] \in \mathbb{R}, B=[b_{ij}] \in \mathbb{R} $$
\end{Theorem*}
\subsubsection*{Metodo Cramer}
\begin{multicols}{2}