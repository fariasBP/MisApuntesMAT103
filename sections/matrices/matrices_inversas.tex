\subsection{Matrice inversa I}
Una matriz inversa se define de la siguiente manera:
\begin{Theorem*} {Matriz inversa}
	Se dice que una matriz cuadrada $A$ es inversa (o invertible, o no singular) si existe una matriz $B$ tal que:
	$$
	AB = BA = I
	$$
\end{Theorem*}
Siendo I la matriz identidad, entonces denominamos $B$ la inversa de $A$ y la denotamos por $A^{-1}$, así también $A$ es la inversa de $B$, $B^{-1}$; por lo tanto:
$$
	AA^{-1}=A^{-1}A=I
$$
La forma para hallar la inversa, se reduce la matriz a una identidad, y una identidad transformarla con las mismas operaciones elementales que se realizo al reducir la matriz.\\\\
\addtocounter{exr}{1} 
\colorbox{gray!55}{\textcolor{white}{Ej.\arabic{exr}) matriz inversa}}
	Hallar $A^{-1}$ y verificar el teorema de la matriz inversa, si: 
	$
	A=\left[\begin{smallmatrix}
		2&-2&2\\
		2&1&0\\
		3&-2&2
	\end{smallmatrix}\right]
	$
\textcolor{gray}{Solución }
	Trabajaremos con dos matrices a la vez una, la matriz A y la matriz identidad, nuestro estrategia será transformar la matriz A en identidad con operaciones pero afectando a la matriz identidad, de la siguiente forma:
	\begin{align*}
		&\begin{array}{l|lr}
			\begin{matrix}
				2&-2&2\\
				2&1&0\\
				3&-2&2
			\end{matrix} & \begin{matrix}
				1&0&0\\
				0&1&0\\
				0&0&1
			\end{matrix} & \begin{matrix}
				-f_2+f_3\rightarrow f_3
			\end{matrix}\\
			\hline
			\begin{matrix}
				2&-2&2\\
				2&1&0\\
				1&-3&2
			\end{matrix} & \begin{matrix}
				1&0&0\\
				0&1&0\\
				0&-1&1
			\end{matrix} & \begin{matrix}
				-2f_3+f_1\rightarrow f_1\\
				-2f_3+f_2\rightarrow f_2
			\end{matrix}\\
			\hline
			\begin{matrix}
				0&4&-2\\
				0&7&-4\\
				1&-3&2
			\end{matrix} & \begin{matrix}
				1&2&-2\\
				0&3&-2\\
				0&-1&1
			\end{matrix} & \begin{matrix}
				f_3+f_1\rightarrow f_1
			\end{matrix}\\
			\hline
			\begin{matrix}
				0&1&0\\
				0&7&-4\\
				1&-3&2
			\end{matrix} & \begin{matrix}
				1&1&-1\\
				0&3&-2\\
				0&-1&1
			\end{matrix} & \begin{matrix}
				-7f_1+f_2\rightarrow f_2\\
				3f_1+f_3\rightarrow f_3
			\end{matrix}\\
			\hline
			\begin{matrix}
				0&1&0\\
				0&0&-4\\
				1&0&2
			\end{matrix} & \begin{matrix}
				1&1&-1\\
				-7&-4&5\\
				3&2&-2
			\end{matrix} & \begin{matrix}
				-\frac{1}{4}f_2\rightarrow f_2
			\end{matrix}\\
			\hline
			\begin{matrix}
				0&1&0\\
				0&0&1\\
				1&0&2
			\end{matrix} & \begin{matrix}
				1&1&-1\\
				\frac{7}{4}&1&-\frac{5}{4}\\
				3&2&-2
			\end{matrix} & \begin{matrix}
				-2f_2+f_3\rightarrow f_3
			\end{matrix}\\
			\hline
			\begin{matrix}
				0&1&0\\
				0&0&1\\
				1&0&0
			\end{matrix} & \begin{matrix}
				1&1&-1\\
				\frac{7}{4}&1&-\frac{5}{4}\\
				-\frac{1}{2}&0&\frac{1}{2}
			\end{matrix} & \begin{matrix}
				f_3\leftrightarrow f_1\\
				f_3\leftrightarrow f_2
			\end{matrix}\\
			\hline
			\begin{matrix}
				1&0&0\\
				0&1&0\\
				0&0&1
			\end{matrix} & \begin{matrix}
				-\frac{1}{2}&0&\frac{1}{2}\\
				1&1&-1\\
				\frac{7}{4}&1&-\frac{5}{4}
			\end{matrix} & \begin{matrix}
				\
			\end{matrix}
		\end{array}
	\end{align*}
	\begin{align*}
		&\Longrightarrow \ A^{-1}=\begin{bmatrix}
			-1/2&0&1/2\\
			1&1&-1\\
			7/4&1&-5/4
		\end{bmatrix}
	\end{align*}
	verificando:
	\begin{align*}
		& A^{-1}A=AA^{-1}=I\\
		&A^{-1}A=\begin{bmatrix}
			-1/2&0&1/2\\
			1&1&-1\\
			7/4&1&-5/4
		\end{bmatrix}\begin{bmatrix}
			2&-2&2\\
			2&1&0\\
			3&-2&2
		\end{bmatrix}\\
		&A^{-1}A=\begin{bmatrix}
			1&0&0\\
			0&1&0\\
			0&0&1
		\end{bmatrix}=I\\
		&AA^{-1}=\begin{bmatrix}
			2&-2&2\\
			2&1&0\\
			3&-2&2
		\end{bmatrix}\begin{bmatrix}
			-1/2&0&1/2\\
			1&1&-1\\
			7/4&1&-5/4
		\end{bmatrix}\\
		&AA^{-1}=\begin{bmatrix}
			1&0&0\\
			0&1&0\\
			0&0&1
		\end{bmatrix}=I\\
		&\therefore \ \text{Se verifica}
	\end{align*}
\hspace*{\fill}\colorbox{gray!55}{ }