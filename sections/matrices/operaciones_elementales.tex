\subsection{Operaciones elementales (O.E.)}
Son cambios que se pueden realizar en las filas y/o columnas de una matriz con el objetivo de encontrar una matriz equivalente
\begin{enumerate}
	\item Intercambiar dos filas o columnas
	\begin{gather*}
		f_i\leftrightarrow f_j \\
		c_i\leftrightarrow c_j
	\end{gather*}
	\item Multiplicar por un escalar una fila o columna no nulo
	\begin{gather*}
		kf_i\rightarrow f_i \\
		kc_i\rightarrow c_i
	\end{gather*}
	\item suma una fila o columna k veces a otra
	\begin{gather*}
		kf_i+f_j\rightarrow f_j \\
		kc_i+c_j\rightarrow c_j
	\end{gather*} 
\end{enumerate}
\addtocounter{exr}{1} 
\colorbox{gray!55}{\textcolor{white}{Ej.\arabic{exr}) O.E. reduc
		ción a escalonada}}
	Reducir $A$, a una forma escalonada
	$$ A=\begin{bmatrix}
		1 & 2 & -3 & 0 \\
		2 & 4 & -2 & 2 \\
		3 & 6 & -4 & 3
	\end{bmatrix} $$
\textcolor{gray}{Solución }
	\begin{align*}
		&\begin{array} {c}
			=\begin{bmatrix}
				1 & 2 & -3 & 0 \\
				2 & 4 & -2 & 2 \\
				3 & 6 & -4 & 3
			\end{bmatrix} \\
			-2f_1+f_2\rightarrow f_2
		\end{array}
		\begin{array} {c}
			\sim\begin{bmatrix}
				1 & 2 & -3 & 0 \\
				0 & 0 & 4 & 2 \\
				3 & 6 & -4 & 3
			\end{bmatrix} \\
			-3f_1+f_3\rightarrow f_3
		\end{array} \\
		&\begin{array} {c}
			\sim\begin{bmatrix}
				1 & 2 & -3 & 0 \\
				0 & 0 & 4 & 2 \\
				0 & 0 & 5 & 3
			\end{bmatrix} \\
			4f_3\rightarrow f_3
		\end{array}
		\begin{array} {c}
			\sim\begin{bmatrix}
				1 & 2 & -3 & 0 \\
				0 & 0 & 4 & 2 \\
				0 & 0 & 20 & 12
			\end{bmatrix} \\
			-5f_2+f_3\rightarrow f_3
		\end{array} \\
		&\begin{array} {c}
			A\sim\begin{bmatrix}
				1 & 2 & -3 & 0 \\
				0 & 0 & 4 & 2 \\
				0 & 0 & 0 & 2
			\end{bmatrix}
		\end{array}
	\end{align*}
\hspace*{\fill}\colorbox{gray!55}{ } \\
\subsubsection*{Consecuencia de operaciones elementales}
Cuando se trata un matriz con operaciones elementales, se obtiene una matriz distinta a la original y esta puede ser matriz elemental o equivalente: \\
\textbf{\textit{Matriz elemental}}, es aquella que se obtiene al aplicar una sola operación elemental sobre una matriz identidad.
\begin{align*}
	&\begin{array} {c}
		\begin{bmatrix}
			1&0\\
			0&1
		\end{bmatrix}\\
		2f_1\rightarrow f_1
	\end{array}\sim\begin{bmatrix}
		2&0\\
		0&1
	\end{bmatrix}=E=\text{matriz elemental}
\end{align*}
\textbf{\textit{Matriz equivalente}}, es aquella que se obtiene al aplicar una o mas operaciones elementales a la original.
\begin{align*}
	&A=\begin{array} {c}
		\begin{bmatrix}
			1&2\\
			4&0
		\end{bmatrix}\\
		f_1\leftrightarrow f_2
	\end{array}\sim\begin{bmatrix}
		4&0\\
		1&2
	\end{bmatrix}=B\\
	&\Rightarrow A\sim B (\text{A  y B son equivalentes})
\end{align*}
\textbf{\textit{matrices mas comunes}} \\
Cuando se realiza operaciones elementales a una matriz, casi siempre se el objetivo sera llegar a las siguientes matrices:
\begin{itemize}
	\item Matriz triangular superior o forma escalonada
	$$
	\begin{bmatrix}
		2&7&0\\
		0&5&6\\
		0&0&1
	\end{bmatrix} \text{ ó } \begin{bmatrix}
		1&2&5&3\\
		0&5&9&1\\
		0&0&3&1
	\end{bmatrix}
	$$
	\item  Matriz triangular inferior
	$$
	\begin{bmatrix}
		1&0&0\\
		5&-1&0\\
		3&2&3
	\end{bmatrix} \text{ ó } \begin{bmatrix}
		2&0&0\\
		5&-4&0\\
		8&3&1\\
		9&2&8
	\end{bmatrix}
	$$
	\item Forma escalonada reducida (Matriz identidad)\\
	una matriz a la cual se quiere llegar siempre es la matriz identidad, debido que es la forma mas comun de hallar la inversa, es decir si se llega a la identidad entonces existe la inversa de la matriz que se transformó
	$$
	\begin{bmatrix}
		1&0&0\\
		0&1&0\\
		0&0&1
	\end{bmatrix}
	$$
	sin embargo si esto no es posible y se llega a algo como esto:
	$$
	\begin{bmatrix}
		1&0&0\\
		0&1&0\\
		0&0&0
	\end{bmatrix}
	$$
	entonces concluimos que la matriz no tiene inversa
\end{itemize}
\subsubsection{Estrategias para reducir y transformación matrices}
Las operaciones elementales son útiles para transformar a una matriz equivalente deseada. \\
\textbf{\textit{Transformar a una matriz identidad equivalente}}
\begin{enumerate}
	\item \textit{Se busca el pivote}, en la primera columna, esta debe ser 1 sino se debe realizar una operación elemental para obtenerlo.
	\item \textit{Crear ceros}, a partir del pivote en la columna en la que se trabaja.
	\item \textit{Cambiar a la siguiente columna}, y buscar el pivote que ahora siempre debe ser el que tiene mas ceros a la izquierda, el pivote debe ser 1 sino realizar una operación elemental para obtenerlo. Realizar el paso 2.
	\item \textit{Repetir}, el paso 3 hasta que se llegue a la ultima columna (nota: si se trabaja en las primera columnas de la izquierda abra varios pivotes, escoger cualquier o el mas conveniente; a medida que se llegue a la ultima columna el numero de pivotes disminuye has ser uno solo).
	\item \textit{Ordenar}, para que tenga la forma de una matriz identidad
\end{enumerate}
\addtocounter{exr}{1} 
\colorbox{gray!55}{\textcolor{white}{Ej.\arabic{exr}) O.E. reducción a identidad}}
	Reducir a identidad:
	$$
	A=\begin{bmatrix}
		3 & 2 & 0 \\
		4 & 1 & 2 \\
		0 & 4 & 1
	\end{bmatrix}
	$$
\textcolor{gray}{Solución }
	\begin{flalign*}
		&=\begin{bmatrix}
			3 & 2 & 0 \\
			4 & 1 & 2 \\
			0 & 4 & 1
		\end{bmatrix} \begin{matrix}
			(\text{creando pivote en} \\
			\text{columna 1}) \\
			-f_1+f_2\rightarrow f_2
		\end{matrix} \\
		&\sim\begin{bmatrix}
			3 & 2 & 0 \\
			\framebox[0.5cm]{1} & -1 & 2 \\
			0 & 4 & 1
		\end{bmatrix}\begin{matrix}
			(\text{creando ceros en}\\
			\text{columna 1})\\
			-3f_2+f_1\rightarrow f_1
		\end{matrix} \\
		&\sim\begin{bmatrix}
			0 & 5 & -6 \\
			1 & -1 & 2 \\
			0 & 4 & 1
		\end{bmatrix}\begin{matrix}
			(\text{Creando pivote en}\\
			\text{columna 2})\\
			-f_3+f_1\rightarrow f_1
		\end{matrix} \\
		&\sim\begin{bmatrix}
			0 & \framebox[0.5cm]{1} & -7 \\
			1 & -1 & 2 \\
			0 & 4 & 1
		\end{bmatrix}\begin{matrix}
			(\text{Creando ceros en}\\
			\text{columna 2})\\
			f_1+f_2\rightarrow f_2\\
			-4f_1+f_3->f_3
		\end{matrix} \\
		&\sim\begin{bmatrix}
			0 & 1 & -7 \\
			1 & 0 & -5 \\
			0 & 0 & -10
		\end{bmatrix}\begin{matrix}
			(\text{Creando pivote en}\\
			\text{columna 3})\\
			-\frac{1}{10}f_3\rightarrow f_3
		\end{matrix} \\
		&\sim\begin{bmatrix}
			0 & 1 & -7 \\
			1 & 0 & -5 \\
			0 & 0 & \framebox[0.5cm]{1}
		\end{bmatrix}\begin{matrix}
			(\text{Creando ceros en}\\
			\text{columna 3})\\
			7f_3+f_1\rightarrow f_1\\
			5f_3+f_2\rightarrow f_2
		\end{matrix} \\
		&\sim\begin{bmatrix}
			0 & 1 & 0 \\
			1 & 0 & 0 \\
			0 & 0 & 1
		\end{bmatrix}\begin{matrix}
			(\text{ordenando})\\
			f_1\leftrightarrow f_2
		\end{matrix} \\
		&\therefore  A\sim\begin{bmatrix}
			1 & 0 & 0 \\
			0 & 1 & 0 \\
			0 & 0 & 1
		\end{bmatrix}
	\end{flalign*}
\hspace*{\fill}\colorbox{gray!55}{ } \\
\textbf{\textit{Reducccion a un matriz escalonada equivalente}} \\
Los pasos son parecidos a la anterior reducción, solo cambia la forma de encontrar el pivote y ya no cambia todas las entradas a ceros 
\begin{enumerate}
	\item \textit{Se busca el pivote}, esta siempre debe estar en la primera columna y en la primera fila, y esta puede ser cualquier numero excepto el cero.
	\item \textit{Crear ceros}, a partir del pivote en la columna en la que se trabaja, estos ceros deben estar estar por debajo del pivote.
	\item \textit{Cambiar a la siguiente columna}, y buscar el pivote que debe estar en la misma fila o una fila por debajo del anterior pivote.
	\item \textit{Repetir}, el paso 3 hasta que se llegue a la ultima columna
\end{enumerate}
\addtocounter{exr}{1} 
\colorbox{gray!55}{\textcolor{white}{Ej.\arabic{exr}) O.E. reducir a escalonada}}
	Reducir $A$ a escalonada
	$$ A=\begin{bmatrix}
		7&4&1&0 \\
		3&2&0&1 \\
		3&3&2&2
	\end{bmatrix} $$
\textcolor{gray}{Solución }
	\begin{flalign*}
		&=\begin{bmatrix}
			7&4&1&0 \\
			3&2&0&1 \\
			4&3&2&2
		\end{bmatrix} \begin{matrix}
			(\text{buscar pivote en} \\
			\text{columna 1})
		\end{matrix} \\
		&\sim\begin{bmatrix}
			\framebox[0.5cm]{7}&4&1&0 \\
			3&2&0&1 \\
			4&3&2&2
		\end{bmatrix} \begin{matrix}
			(\text{Crear ceros por debajo} \\
			\text{del pivote en columna 1}) \\
			-\frac{3}{7}f_1+f_2\rightarrow f_2\\
			-\frac{4}{7}f_1+f_3\rightarrow f_3
		\end{matrix} \\
		&\sim\begin{bmatrix}
			7&4&1&0 \\
			0&2/7&-3/7&1 \\
			0&5/7&10/7&2
		\end{bmatrix} \begin{matrix}
			(\text{Buscar el pivote} \\
			\text{en columna 2})
		\end{matrix} \\
		&\sim\begin{bmatrix}
			7&4&1&0 \\
			0&\framebox[0.8cm]{2/7}&-3/7&1 \\
			0&5/7&10/7&2
		\end{bmatrix} \begin{matrix}
			(\text{Crear ceros por} \\
			(\text{encima del pivote} \\
			\text{en columna 2}) \\
			-\frac{5}{2}f_2+f_3\rightarrow f_3
		\end{matrix} \\
		&\sim\begin{bmatrix}
			7&4&1&0 \\
			0&2/7&-3/7&1 \\
			0&0&5/2&-1/2
		\end{bmatrix} \begin{matrix}
			(\text{Buscar el pivote} \\
			\text{en columna 3})
		\end{matrix} \\
		&\therefore A\sim\begin{bmatrix}
			7&4&1&0 \\
			0&2/7&-3/7&1 \\
			0&0&5/2&-1/2
		\end{bmatrix}
	\end{flalign*}
\hspace*{\fill}\colorbox{gray!55}{ } \\
\subsubsection{Relación de semejanza de matrices}
Se sabe que si se realiza operaciones elementales a $A$ se obtiene $B$ y se cumple que $A\sim B$; note que $A$ es diferente a $B$ ($A\ne B$), entonces la pregunta ¿existe una relacion matricial entre ambos?, y la respuesta es si:
\begin{Theorem*} {}
	Sea $B$ una matriz obtenido mediante operaciones elementales a $A$:
	$$A_{m\cross n}\sim B_{m\cross n}$$
	por lo tanto existe una relación de equivalencia $P$ y $Q$, entre $A$ y $B$ de la forma:
	$$ P_{m\cross m}\cdot A_{m\cross n}\cdot Q_{n\cross n}=B_{m\cross n} $$
\end{Theorem*}
donde $P$ es el producto de matrices elementales en fila y $Q$ es el producto de matrices elementales en columna, es decir: \\
\begin{gather*}
	P=E_n\cdot E_{n-1}\cdot E_{n-2}\cdot...\cdot E_2\cdot E_1\\
	Q=D_1\cdot D_2\cdot D_3\cdot...\cdot D_{n-1}\cdot D_n
\end{gather*}
Ademas:
$$
	E_0=\left[\begin{smallmatrix}
		1&0&0&\cdots&0\\
		0&1&0&\cdots&0\\
		0&0&1&\cdots&0\\
		\vdots&\vdots&\vdots&\ddots&\vdots\\
		0&0&0&\cdots&1
	\end{smallmatrix}\right] \quad\quad D_0=\left[\begin{smallmatrix}
		1&0&0&\cdots&0\\
		0&1&0&\cdots&0\\
		0&0&1&\cdots&0\\
		\vdots&\vdots&\vdots&\ddots&\vdots\\
		0&0&0&\cdots&1
	\end{smallmatrix}\right]
$$\\
\addtocounter{exr}{1} 
\colorbox{gray!55}{\textcolor{white}{Ej.\arabic{exr}) relación de semejanza de matrices}}
	Hallar la relación Q,Q de las matrices A y B, luego de realizar 3 operaciones elementales en fila y 2 en columna
	$$ A=\begin{bmatrix}
		3&4&1&2\\
		2&0&5&0\\
		1&3&2&1
	\end{bmatrix} $$ \\
\textcolor{gray}{Solución }\\
Se realiza 3 y 2, operaciones elementales en fila y columna cualesquiera, respectivamente:
	\begin{align*}
		&\begin{array} {c}
			A=\begin{bmatrix}
				3&4&1&2\\
				2&0&5&0\\
				1&3&2&1
			\end{bmatrix} \\
			f_1\leftrightarrow f_2
		\end{array}
		\begin{array} {c}
			\sim\begin{bmatrix}
				2&0&5&0\\
				3&4&1&2\\
				1&3&2&1
			\end{bmatrix} \\
			3f_3\rightarrow f_3
		\end{array} \\
		&\begin{array} {c}
			=\begin{bmatrix}
				2&0&5&0\\
				3&4&1&2\\
				3&9&6&3
			\end{bmatrix} \\
			2f_1+f_2\rightarrow f_2
		\end{array}
		\begin{array} {c}
			\sim\begin{bmatrix}
				2&0&5&0\\
				3&4&1&2\\
				1&3&2&1
			\end{bmatrix} \\
			c_2\leftrightarrow c_4
		\end{array} \\
		&\begin{array} {c}
			=\begin{bmatrix}
				2&0&5&0\\
				7&2&11&4\\
				3&3&6&9
			\end{bmatrix} \\
			c_1+c_2\rightarrow c_2
		\end{array}
		\begin{array} {c}
			\sim\begin{bmatrix}
				2&2&5&0\\
				7&9&11&4\\
				3&6&6&9
			\end{bmatrix} \\
			\
		\end{array}
	\end{align*}
	entonces:
	\begin{align*}
		&B=\begin{bmatrix}
			2&2&5&0\\
			7&9&11&4\\
			3&6&6&9
		\end{bmatrix}\quad A=\begin{bmatrix}
			3&4&1&2\\
			2&0&5&0\\
			1&3&2&1
		\end{bmatrix}
	\end{align*}
	se sabe que:
	\begin{align*}
		&A_{3\cross 4}\sim B_{3\cross 4} \Rightarrow \exists \ P, Q \\
		&\text{tal que: }P_{3\cross3}A_{3\cross4}Q_{4\cross4}=B_{3\cross4}
	\end{align*}
	para P se hicieron 3 operaciones elementales:
	\begin{align*}
		&\text{en filas:}\\
		&\text{1)}f_1\leftrightarrow f_2\\
		&\text{2)}3f_3\rightarrow f_3\\
		&\text{3)}2f_1+f_2\rightarrow f_2\\
		&\Rightarrow P_{3\cross3}=E_3\cdot E_2\cdot E_1 \Leftrightarrow E_0=\left[\begin{smallmatrix}
			1&0&0\\
			0&1&0\\
			0&0&1
		\end{smallmatrix}\right]\\
		&\begin{bmatrix}
			1&0&0\\
			0&1&0\\
			0&0&1
		\end{bmatrix} f_1\leftrightarrow f_2 \Rightarrow E_1= \begin{bmatrix}
			0&1&0\\
			1&0&0\\
			0&0&1
		\end{bmatrix}\\
		&\begin{bmatrix}
			1&0&0\\
			0&1&0\\
			0&0&1
		\end{bmatrix} 3f_3\rightarrow f_3 \Rightarrow E_2= \begin{bmatrix}
			1&0&0\\
			0&1&0\\
			0&0&3
		\end{bmatrix}\\
		&\begin{bmatrix}
			1&0&0\\
			0&1&0\\
			0&0&1
		\end{bmatrix} 2f_1+f_2\rightarrow f_2 \Rightarrow E_3= \begin{bmatrix}
			1&0&0\\
			2&1&0\\
			0&0&1
		\end{bmatrix}
	\end{align*}
	para Q se hicieron 2 operaciones elementales en columnas:
	\begin{align*}
		&\text{1)}c_2\leftrightarrow c_4\\
		&\text{2)}c_1+c_2\rightarrow c_2\\
		&\Rightarrow Q_{4\cross4}=D_1\cdot D_2 \Leftrightarrow D_0=\left[\begin{smallmatrix}
			1&0&0&0\\
			0&1&0&0\\
			0&0&1&0\\
			0&0&0&1
		\end{smallmatrix}\right]\\
		&\begin{bmatrix}
			1&0&0&0\\
			0&1&0&0\\
			0&0&1&0\\
			0&0&0&1
		\end{bmatrix} c_2\leftrightarrow c_4 \\ &\Rightarrow D_1= \begin{bmatrix}
			1&0&0&0\\
			0&0&0&1\\
			0&0&1&0\\
			0&1&0&1
		\end{bmatrix}\\
		&\begin{bmatrix}
			1&0&0&0\\
			0&1&0&0\\
			0&0&1&0\\
			0&0&0&1
		\end{bmatrix} c_1+c_2\rightarrow c_2 \\ &\Rightarrow D_2= \begin{bmatrix}
			1&1&0&0\\
			0&1&0&0\\
			0&0&1&0\\
			0&0&0&1
		\end{bmatrix}
	\end{align*}
	finalmente:
	\begin{align*}
		&P=E_3\cdot E_2\cdot E_1 \\
		&P=\begin{bmatrix}
			1&0&0\\
			2&1&0\\
			0&0&1
		\end{bmatrix}\begin{bmatrix}
			1&0&0\\
			0&1&0\\
			0&0&3
		\end{bmatrix}\begin{bmatrix}
			0&1&0\\
			1&0&0\\
			0&0&1
		\end{bmatrix}\\
		&P= \begin{bmatrix}
			0&1&0\\
			1&2&0\\
			0&0&3
		\end{bmatrix}\\
		&Q=D_1\cdot D_2 \\
		&Q=\begin{bmatrix}
			1&0&0&0\\
			0&0&0&1\\
			0&0&1&0\\
			0&1&0&1
		\end{bmatrix}\begin{bmatrix}
			1&1&0&0\\
			0&1&0&0\\
			0&0&1&0\\
			0&0&0&1
		\end{bmatrix} \\
		&Q= \begin{bmatrix}
			1&1&0&0\\
			0&0&0&1\\
			0&0&1&0\\
			0&1&0&0
		\end{bmatrix}
	\end{align*}
	verificando:
	\begin{align*}
		&P\cdot A\cdot Q = B\\
		&\begin{bmatrix}
			0&1&0\\
			1&2&0\\
			0&0&3
		\end{bmatrix}\begin{bmatrix}
			3&4&1&2\\
			2&0&5&0\\
			1&3&2&1
		\end{bmatrix}\begin{bmatrix}
			1&1&0&0\\
			0&0&0&1\\
			0&0&1&0\\
			0&1&0&0
		\end{bmatrix}=B\\
		&\begin{bmatrix}
			2&2&5&0\\
			7&9&11&4\\
			3&6&6&9
		\end{bmatrix}=B
	\end{align*}
\hspace*{\fill}\colorbox{gray!55}{ } \\
\subsubsection{Factorización LU}
Es el proceso de transformar una matriz $A$ en dos, una triangular inferior $L$ y otra triangular superior $U$.
$$
	A=LU
$$
Para este fin se dobe optar por los siguientes pasos:
\begin{enumerate}
	\item Reducir la matriz $A$ a una forma triangular superior que llamaremos $U$, por medio de operaciones elementales $iii$.\\
	\item Obtendremos $L$ a partir de las operaciones elementales $iii$, donde los valores serán $k$ con signo cambiado y su diagonal principal sera 1.
\end{enumerate}
\addtocounter{exr}{1} 
\colorbox{gray!55}{\textcolor{white}{Ej.\arabic{exr}) factorización LU}}
	Factorizar $A=\begin{bmatrix}
		1&2&-3\\
		-3&-4&13\\
		2&1&-5
	\end{bmatrix}$
\textcolor{gray}{Solución }
	\begin{align*}
		&=\begin{bmatrix}
			1&2&-3\\
			-3&-4&13\\
			2&1&-5
		\end{bmatrix} \begin{matrix}
			\framebox[0.5cm]{3} \ f_1+f_2\rightarrow f_2\\
			\framebox[0.8cm]{-2} \ f_1+f_3\rightarrow f_3
		\end{matrix}\\
		&\sim\begin{bmatrix}
			1&2&-3\\
			0&2&4\\
			0&-3&1
		\end{bmatrix} \begin{matrix}
			\framebox[0.5cm]{$\frac{3}{2}$} \ f_2+f_3\rightarrow f_3
		\end{matrix}\\
		&\sim\begin{bmatrix}
			1&2&-3\\
			0&2&4\\
			0&0&7
		\end{bmatrix}=U\\
		&\Rightarrow L=\begin{bmatrix}
			1&0&0\\
			\framebox[0.8cm]{-3}&1&0\\
			\framebox[0.8cm]{2}&\framebox[1cm]{-3/2}&1
		\end{bmatrix}\\
		&\therefore \ A=\begin{bmatrix}
			1&0&0\\
			-3&1&0\\
			2&-3/2&1
		\end{bmatrix}\begin{bmatrix}
			1&2&-3\\
			0&2&4\\
			0&0&7
		\end{bmatrix}
	\end{align*}
\hspace*{\fill}\colorbox{gray!55}{ } \\