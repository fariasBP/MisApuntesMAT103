\subsection{Otras operaciones de matrices}
\subsubsection{Matriz transpuesta}
\begin{Theorem*} {Matriz transpuesta}
	La transpuesta de una matriz $A$, denotada por $A^T$, es la matriz obtenida escribiendo las filas de $A$ por orden como columnas:
	$$ A_{m\cross n} = [a_{ij}] \rightarrow A^T_{n\cross m} = [a_{ji}] $$
\end{Theorem*}
Es decir las filas de $A$ pasan a ser columnas de $A^T$ y las columnas de $A$ pasa ser filas de $A^T$. \\\\
\addtocounter{exr}{1} 
\colorbox{gray!55}{\textcolor{white}{Ej.\arabic{exr}) matriz transpuesta}}
	Hallar la transpuesta de $A=\left[\begin{smallmatrix}
		3 & -4 \\
		5 & 6 \\
		2 & 8 
	\end{smallmatrix}\right]$\\
\textcolor{gray}{Solución }
	\begin{flalign*}
		A_{3\cross2}=\begin{bmatrix}
			3 & -4 \\
			5 & 6 \\
			2 & 8
		\end{bmatrix} \\
		\therefore A^T_{2\cross3}=\begin{bmatrix}
			3 & 5 & 2 \\
			-4 & 6 & 8
		\end{bmatrix}
	\end{flalign*}
\hspace*{\fill}\colorbox{gray!55}{ } \\
\noindent\textbf{\textit{Propiedades de una matriz transpuesta}}
\begin{enumerate}
	\item $ (A+B)^T=A^T+B^T $
	\item $ (A^T)^T=A $
	\item $ (kA)^T=kA^T $ (k: escalar)
	\item $ (AB)^T=A^TB^T $
\end{enumerate}
\subsubsection{Traza de una matriz}
La traza es una operacion donde interviene la diagonal principal
\begin{Theorem*} {Traza de una matriz}
	Sea $A_{n\cross n=[a_{ij}]}$ entonces su traza denotada por $tr(A)$ se define de la siguiente manera:
	$$ tr(A)=\sum_{i=1}^{n}aii $$
\end{Theorem*}
Es decir, que la traza viene siendo la suma de los elementos de la diagonal principal de una matriz cuadrada. \\\\
\addtocounter{exr}{1} 
\colorbox{gray!55}{\textcolor{white}{Ej.\arabic{exr}) traza de una matriz}}
	Hallar la traza de $A$
	$$A=\begin{bmatrix}
		2 & 1 & 3 \\
		4 & 2 & 5 \\
		0 & 0 & -1
	\end{bmatrix}$$
\textcolor{gray}{Solución }
	\begin{flalign*}
		&A=\begin{bmatrix}
			2 & 1 & 3 \\
			4 & 2 & 5 \\
			0 & 0 & -1
		\end{bmatrix} \\
		&tr(A)=2+2+(-1) \\
		&\therefore tr(A)=3
	\end{flalign*}
\hspace*{\fill}\colorbox{gray!55}{ }