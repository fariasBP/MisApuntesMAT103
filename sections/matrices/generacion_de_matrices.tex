\subsection{Generación de matrices}
\subsubsection{Matrices cuadradas}
\begin{flalign*}
	&A_{m\cross m} = \begin{bmatrix}
		a_{11} & a_{12} & \cdots & a_{1n} \\
		a_{21} & a_{22} & \cdots & a_{2n} \\
		\vdots & \vdots & \ddots & \vdots \\
		a_{m1} & a_{m2} & \cdots & a_{mn} \\
	\end{bmatrix}
\end{flalign*}
\subsubsection{Matriz delta de Kronecker}
\begin{flalign*}
	&\delta_{n\cross n}=\left[ \ \ \delta_{ij}=\left\{\begin{matrix}
		d & \text{si} & i=j \\
		0 & \text{si} & i\ne j
	\end{matrix}\right. \ \ \right] \\
	&\delta_{n\cross n}=\begin{bmatrix}
		d_{11} & 0 & 0 & \cdots & 0 \\
		0 & d_{22} & 0 & \cdots & 0 \\
		0 & 0 & d_{33} & \cdots & 0 \\
		\vdots & \vdots & \vdots & \ddots & \vdots \\
		0 & 0 & 0 & \cdots & d_{nn}
	\end{bmatrix}
\end{flalign*}
Un caso especial es la matriz identidad:
\begin{flalign*}
	&I_{n\cross n}=\left[ \ \ \delta_{ij}=\left\{\begin{matrix}
		1 & \text{si} & i=j \\
		0 & \text{si} & i\ne j
	\end{matrix}\right. \ \ \right] \\
	&I_{n\cross n}=\begin{bmatrix}
		1 & 0 & 0 & \cdots & 0 \\
		0 & 1 & 0 & \cdots & 0 \\
		0 & 0 & 1 & \cdots & 0 \\
		\vdots & \vdots & \vdots & \ddots & \vdots \\
		0 & 0 & 0 & \cdots & 1
	\end{bmatrix}
\end{flalign*}
\subsubsection{Matriz nulo}
\begin{flalign*}
	&0_{m\cross n}=\left[ \ \ 0_{ij}=0 \ \ \right] \\
	&0_{m\cross n}=\begin{bmatrix}
		0 & 0 & 0 & \cdots & 0 \\
		0 & 0 & 0 & \cdots & 0 \\
		0 & 0 & 0 & \cdots & 0 \\
		\vdots & \vdots & \vdots & \ddots & \vdots \\
		0 & 0 & 0 & \cdots & 0
	\end{bmatrix}
\end{flalign*}